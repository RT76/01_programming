\documentclass[UTF8]{ctexart}
% \usepackage{float}
\usepackage{mathtools}
\usepackage{amsmath}
\usepackage{cases}

\title{数学大作业}
\author{宋和德}
\date{\today}

\begin{document}
\maketitle

\part{选择题}
1.C 2.B 3.A 4.A 5.D
\part{计算题}
\section{}
将$f(x)$进行偶延拓可得:$b_n=0$,计算傅里叶系数得:
\[\qquad a_0=\frac{2}{\pi}\int_0^\pi x \mathrm{d} x=\pi\]
% \[a_n=\frac{2}{\pi}\int_0^\pi x\cos(nx) \mathrm{d} x=\frac{2}{\pi}\frac{(-1)^2-1}{n^2}\]

\begin{equation}
    a_n=\frac{2}{\pi}\int_0^\pi x\cos(nx) \mathrm{d} x=\frac{2}{\pi}\frac{(-1)^n-1}{n^2}=\begin{cases}
        0,                 & n=2k   \\
        -\frac{4}{n^2\pi}, & n=2k+1
    \end{cases}
\end{equation}
由(1)可得:
\[f(x)=\frac{\pi}{2}+\sum_{k=0}^\infty-\frac{4}{(2k+1)^2\pi}\cos(2k+1)x
    =\frac{\pi}{2}-\frac{4}{(2k+1)^2\pi}\sum_{k=0}^\infty\cos(2k+1)x\]\\
\[=\frac{\pi}{2}-\frac{4}{\pi}(\cos x+\frac{\cos3x}{3^2}+\frac{\cos5x}{5^2}+\cdots+\frac{\cos(2k+1)x}{n^2}+\cdots)\]
\section{}
对$x$求导得:\\
\[\left\{\begin{lgathered}
        2x+2y\frac{\mathrm{d}y}{\mathrm{d}x}=0\\
        2x+2z\frac{\mathrm{d}z}{\mathrm{d}x}=0\\
    \end{lgathered}\right.\]
解得:
\[\left\{\begin{lgathered}
        \frac{\mathrm{d}y}{\mathrm{d}x}=-\frac{x}{y}\\
        \frac{\mathrm{d}z}{\mathrm{d}x}=-\frac{x}{z}\\
    \end{lgathered}\right.\]
在点$M (\frac{R}{\sqrt{2}},\frac{R}{\sqrt{2}},\frac{R}{\sqrt{2}})$处切向量$\vec{T} =(1,-1,-1)$\\

切线方程:\[\frac{x-\frac{R}{\sqrt{2}}}{2R^2}=-\frac{y-\frac{R}{\sqrt{2}}}{2R^2}=-\frac{z-\frac{R}{\sqrt{2}}}{2R^2}\]\\
法平面方程:\[x-y-z=-\frac{R}{\sqrt{2}}\]
\section{}
由一元函数泰勒展开式得:\\
\[\sin(x+y)=(x+y)-\frac{(x+y)^3}{3!}+o(x+y)^3;\]
\section{}
易知\[
    f(x)=\lim_{n \to \infty}  f_n(x)=0\]取$x_n=\frac{1}{n}\in (0,1) $,则有:\[\beta_n\geq|f_n(x_n)-f(x_n)|=\frac{1}{2}\geq\varepsilon\]
故$f_n(x)$在$(0,1)$上不一致收敛。
\section{}
\[\frac{\partial   z }{\partial  y }=f'(\frac{\partial  z }{\partial  y }+x)\Rightarrow \frac{\partial  z }{\partial  y }=\frac{f'x}{1-f'}\]
同理可得:\[\frac{\partial  z }{\partial  x }=\frac{f'x}{1-f'}\]
\[\frac{\partial^2  z }{\partial x \partial  y }=\frac{\partial}{\partial x}(\frac{\partial  z }{\partial  y })=\frac{\partial}{\partial x}(\frac{f'x}{1-f'})=\frac{(1-f')[f'+xf''(y+\frac{\partial  z }{\partial  y })]+f'f''x(y+\frac{\partial  z }{\partial  x })}{(1-f')^2}\\\]
将$\frac{\partial   z }{\partial  y },\frac{\partial  z }{\partial  x }$代入并化简得:
\[\frac{\partial^2  z }{\partial x \partial  y }=\frac{f'(1-f')^2+f''xy}{(1-f')^3}\]
\part{证明不等式}
当$x=y=0$时,不等式显然成立。当$x>0,y>0$时,设$x+y=a>0$;\\
问题转化为求$f(x,y)=\frac{x^n+y^n}{2}$在约束条件$x+y=a$下的最值。\\
构造Lagrange函数:\[
    L(x,y,\lambda )=\frac12(x^n+y^n)-\lambda(x+y-a)\]
求解方程组
\[
    \left\{\begin{lgathered}
        \frac{\partial L}{\partial x}=\frac{n}{2}x^{n-1}+\lambda=0\\
        \frac{\partial L}{\partial y}=\frac{n}{2}y^{n-1}+\lambda=0\\
        x+y=a\\
    \end{lgathered}\right.\]
得唯一极值点$P (\frac a 2,\frac{a}{2})$,
验算得$P $点Hessian矩阵$H (P)$为正定矩阵,故$P $点为极小值点。
将$P $点函数值与边界点$(0,a),(a,0)$函数值比较得:
\[
    f(x,y)\geq f(\frac{a}{2},\frac{a}{2})=f(0,a)=f(a,0)=(\frac{a}{2})^n=(\frac{x+y}{2})^n\]
即证得:\[\frac{x^n+y^n}{2}\geq (\frac{x+y}{2})^n\qquad (x>0,y>0)\]
\part{证明题}
记\[u_n(x)=\frac{\ln(1+nx)}{n^3}\]则\[f(x)=\sum_{n=1}^\infty u_n(x)\]\\
\[u'_n(x)=\frac{1}{n^2(1+nx)}\]
显然$\sum_{n=1}^\infty u'_n(x)$在$x\in (0,1)$一致收敛\\
注意到:
\[f(x)<\sum_{n=1}^\infty\frac{\ln (1+n)}{n^3}\leq \sum_{n=1}^\infty\frac{1}{n^2}\]
故$f(x)=\sum_{n=1}^\infty u_n(x)$在$x\in (0,1)$收敛;
由函数项级数的逐项可导定理,在$x\in (0,1)\quad f(x)$一致收敛到一个可导函数$f'(x)$
\part{}
(1)因为:
\[\lim_{(x,y)  \to  (0,0)}f(x,y)=0=f(0,0)\]
所以$f(x,y)$在$(0,0)$点连续。\\\\
\phantom{空格}(2)
\[
    f_x(0,0)=\lim_{\Delta x  \to  0}\frac{f(\Delta x,0)-f(0,0)}{\Delta x}=\lim_{\Delta x  \to  0}\frac{(\Delta x)^3}{(\Delta x)^3}=1\]
由$x$与$y$的轮换对称性可得\[f_y(0,0)=1\]
接下来考察极限\[
    \lim_{(\Delta x,\Delta y)  \to  (0,0)}\frac{f(\Delta x,\Delta y)-f_x(0,0)\Delta x-f_y(0,0)\Delta y-f(0,0)}{\sqrt{x^2+y^2}}\]
的存在性及其值。
\[
    \lim_{(\Delta x,\Delta y)  \to  (0,0)}\frac{f(\Delta x,\Delta y)-f_x(0,0)\Delta x-f_y(0,0)\Delta y-f(0,0)}{\sqrt{x^2+y^2}}\]
\[=\lim_{(\Delta x,\Delta y) \to (0,0)}-\frac{\Delta x(\Delta y)^2+\Delta y(\Delta x)^2}{[(\Delta x)^2+(\Delta y)^2]^{\frac{3}{2}}}\]
\[ =\lim_{(\Delta x,\Delta y) \to  (0,0),y=kx}\frac{k^2+k}{(k^2+1)^{\frac{3}{2}}}\]
故极限不存在,所以$f(x,y)$在$(0,0)$处不可微。由于函数在$(0,0)$不可微,故至少有一个偏导数在$(0,0)$不连续;又由$x$与$y$的轮换对称性,$f_x(x,y),f_y(x,y)$在$(0,0)$都不连续\\\\
\phantom{空格}(3)方向$(1,1)$的单位向量$\vec n=(\frac{\sqrt 2}{2},\frac{\sqrt 2}{2})$故沿其方向的方向导数
\[\frac{\partial f}{\partial \vec{n}}=\lim_{ t  \to  0^+}\frac{f(0+\frac{\sqrt{2}}{2}t,0+\frac{\sqrt{2}}{2}t)}{t}=\lim_{ t  \to  0^+}\frac{\frac{\sqrt{2}t^3}{2}}{t^3}=\frac{\sqrt{2}}{2}\]
\part{}
由Cauchy-Hadamard公式,收敛半径
\[R=\frac{1}{\varlimsup_{n  \to  \infty}\sqrt[n]{|n(n-1)|}}=1\]
当$x=-1$时,原级数为Leibniz级数,收敛;$x=1$时,由裂项相消求和易得级数收敛。
故收敛域$x\in[-1,1]$\\
记和函数:\[f(x)=\sum_{n=2}^\infty\frac{x^n}{n(n-1)}=\sum_{n=1}^\infty\frac{x^{n+1}}{n(n+1)}\]\\
则:\[f'(x)=\sum_{n=1}^\infty\frac{x^n}{n}=-\ln(1-x)\qquad x\in[-1,1)\]\\
所以:
\[f(x)-f(0)=\int_0^xf'(t)\mathrm{d}t=(1-x)\ln(1-x)+x\qquad x\in[-1,1)\]\\
所以:\[f(x)=(1-x)\ln(1-x)+x\]
$x=1$时,
\[f(x)=\sum_{n=1}^\infty\frac{1}{n(n+1)}=\sum_{n=1}^\infty\frac{1}{n}-\frac{1}{n+1}=\lim_{n  \to  \infty}1-\frac{1}{(n+1)}=1\]
综上所述:
\[f(x)=\left\{ \begin{lgathered}
        (1-x)\ln(1-x)+x ;\qquad\qquad x\in[-1,1)\\
        \phantom{空格空格空格空格}1 ;\phantom{空格spacespace空格空格} x=1\\
    \end{lgathered}\right.\]
\part{}
(1)注意到:
\[\sum_{n=1}^\infty\frac{\sin nx\sin x}{\sqrt {n+x}}= \sum_{n=1}^\infty\frac{\cos(n-\frac12)x-\cos(n+\frac12)x}{2\sqrt {n+x}}\]\\
而$\frac{1}{\sqrt{n+x}}$单调一致收敛于0;\[\sum_{n=1}^\infty[\cos(n-\frac12)x-\cos(n+\frac12)x]=\cos\frac12 x-\cos(n+\frac12)x\]一致有界,
由Dirichlet判别法可知级数在$x \in (0,\pi)$上收敛。\\
又因为
\[|\sum_{n=1}^\infty\frac{\sin nx\sin x}{\sqrt {n+x}}|\geq  \sum_{n=1}^\infty\frac{\sin^2 nx|\sin x|}{\sqrt {n+x}}=\sum_{n=1}^\infty\frac{(1-\cos 2nx)|\sin x|}{2\sqrt {n+x}}=\sum_{n=1}^\infty\frac{1}{2\sqrt {n+x}}+\sum_{n=1}^\infty\frac{\cos 2nx|\sin x|}{2\sqrt {n+x}}\]
第一个级数发散,第二个级数收敛,故原级数发散。\\
综上所述:级数\[\sum_{n=1}^\infty\frac{\sin nx\sin x}{\sqrt {n+x}}\]条件收敛。\\\\
\phantom{空格}(2)由(1)可得原级数\[\sum_{n=1}^\infty\frac{\sin nx\sin x}{\sqrt {n+x}}\]
在$x \in (0,\pi)$上内闭一致收敛,故级数\\
\[\sum_{n=1}^\infty\frac{\sin nx\sin x}{\sqrt {n+x}}\]在$x \in (0,\pi)$上连续。
\end{document}