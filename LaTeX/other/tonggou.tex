\documentclass[UTF8,a4paper]{ctexart}
\usepackage{amsmath}
\pagestyle{plain}
\linespread{1.5}
\title{\bfseries{\zihao{1}\heiti{同构在导数中的应用}}}
\author{\kaishu{宋和德}{ }整理}
\date{\today}

\begin{document}
\maketitle
\begin{abstract}
    在导数的一些题目中,常常会遇到双变量的不等式证明或者
    双变量不等式恒成立求参数范围问题。同构是一种通过变形
    将两个变量分开并构造相同结构从而将问题简化为单变量函
    数单调性问题的转化思想。本文整理了常用的一些同构技巧,
    并在文末附有相应的习题加以巩固。
\end{abstract}
\section*{预备知识}

\[
    xe^x=e^{\ln x}e^x=e^{x+\ln x}\qquad\frac{x}{e^x}=e^{\ln x-x}\qquad\frac{e^x}{x}=e^{x-\ln x}
\]
\[
    x+\ln x=\ln e^x+\ln x=\ln xe^x\qquad x-\ln x=\ln \frac{e^x}{x}
\]
\section{抽象函数}
\qquad注意到变形时的变号问题,以下都默认$x_1>x_2$\\
(1)
\[\frac{f(x_1)-f(x_2)}{x_1-x_2} > k\]
\[
    \Rightarrow f(x_1)-f(x_2)>k(x_1-x_2)\Rightarrow
    f(x_1)-kx_1>f(x_2)-x_2
\]
令$h(x)=f(x)-kx$,问题转化为研究$h(x)$单调性。\\
(2)
\[\frac{f(x_1)-f(x_2)}{x_1-x_2}>\frac{k}{x_1x_2}\]
\[
    \Rightarrow f(x_1)-f(x_2)>\frac{k(x_2-x_1)}{x_1x_2}
    =\frac{k}{x_2}-\frac{k}{x_1}\Rightarrow
    f(x_1)+\frac k x_1>f(x_2)+\frac k x_2
\]
令$h(x)=f(x)+\frac k x$,问题转化为研究$h(x)$单调性。\\
\section{指对结构}
指对跨阶想同构,同左同右取对数。\\
(1)积型\begin{equation}
    ae^a\leq b\ln b\Rightarrow \begin{cases}
        \text{同左:} ae^a\leq (\ln b)e^{\ln b}
        \Rightarrow f(x)=xe^x;    \\
        \text{同右:} e^a\ln e^a\leq b\ln b\Rightarrow
        f(x)=x\ln x;              \\
        \text{取对:} a+\ln a\leq\ln b+\ln(\ln b)
        \Rightarrow f(x)=x+\ln x. \\
    \end{cases}
\end{equation}
需要指出的是,取对数是最快的方法,
因为其构造的函数单调性一看便知。\\
(2)商型\begin{equation}
    \frac{e^a}{a}\leq\frac{b}{\ln b}\Rightarrow\begin{cases}
        \text{同左:} \frac{e^a}{a}\leq\frac{e^{\ln b}}{\ln b}
        \Rightarrow f(x)=\frac{e^x}{x};   \\
        \text{同右:} \frac{e^a}{\ln e^a}\leq\frac{b}{\ln b}
        \Rightarrow f(x)=\frac{x}{\ln x}; \\
        \text{取对:}a-\ln a\leq \ln b-\ln(\ln b)\Rightarrow
        f(x)=x-\ln x.
    \end{cases}
\end{equation}
(3)和差型\begin{equation}
    \Longrightarrow
\end{equation}
\end{document}