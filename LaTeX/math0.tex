\documentclass[11pt,space]{ctexart} % ans
\usepackage{GEEexam}
\usepackage{float}
\linespread{1.8}
%\watermark{60}{6}{14-金融工程-白兔兔} %水印

\everymath{\displaystyle}
\begin{document}\zihao{-4}
\juemi %输出绝密
\biaoti{2022年普通高等学校招生全国统一考试}
\fubiaoti{数$\quad$学}
本试卷共4页,22小题,满分150分。考试用时120分钟。\\
{\heiti 注意事项}:
\begin{enumerate}[itemsep=-0.3em,topsep=0pt]
\item 答卷前,考生务必将自己的姓名和准考证号填写在答题卡上。
\item 作答选择题时,选出每小题答案后,用2B铅笔把答题卡对应题目的答案标号涂黑。如需改动,用橡皮擦干净后,再选涂其它答案标号。在试卷上作答无效。
\item 非选择题必须使用黑色字迹的钢笔或签字笔作答,答案必须写在答题卡各题目指定区域内相应位置上。
\item 考生必须保持答题卡的整洁。考试结束后,将本试卷和答题卡一并交回。	
\end{enumerate}
%%====================================================================
%%—————————————————————————————正文开始———————————————————————————————
%%====================================================================

\section{选择题:本大题共8小题,每小题5分,共40分。在每小题给出的四个选项中,只有一项是符合题目要求的。}

\begin{enumerate}[itemsep=0.3em,topsep=0pt]
	\item  若集合$ M=\{x\mid\sqrt{x}<4\}$,  $N=\{x\mid 3x\geqslant 1\} $, 则  $M \cap N=$
	\begin{tasks}(4)
	\task $\{x\mid 0\leqslant x < 2\}$   \task $\{x\mid \frac{1}{3} \leqslant x < 2\}$   
	\task $\{x\mid 3 \leqslant x < 16\}$ \task $\{x\mid \frac{1}{3} \leqslant x < 16\}$
	\end{tasks}
	
	\item  若$\mathrm{i}(1-z)=1$, 则  $z+\bar{z}=$
	\begin{tasks}(4)
	\task $-2$ \task $-1$ \task $1$ \task $2$
	\end{tasks}
	
	\item  在$\triangle ABC$中,点$D$在边$AB$上,$BD=2DA$. 记$\overrightarrow{CA} = \bm{m}$,$\overrightarrow{CD} = \bm{n}$,则$\overrightarrow{CB}=$
	\begin{tasks}(4)
	\task $3\bm{m}-2\bm{n}$ \task $-2\bm{m}+3\bm{n}$ \task $3\bm{m}+2\bm{n}$ \task $2\bm{m}+3\bm{n}$
	\end{tasks}
	
	\item  南水北调工程缓解了北方一些地区水资源短缺问题,其中一部分水蓄入某水库. 已知该水库水位为海拔148.5m时,相应水面的面积为140.0km$^2$;水位为海拔157.5m时,相应水面的面积为180.0km$^2$.将该水库在这两个水位间的形状看作一个棱台,则该水库水位从海拔148.5m上升到157.5m时,增加的水量约为($\sqrt{7}\approx 2.65$)
	\begin{tasks}(4)
	\task $1.0\times 10^9 \mathrm{m}^3$ \task $1.2\times 10^9 \mathrm{m}^3$ \task $1.4\times 10^9 \mathrm{m}^3$ \task $1.6\times 10^9 \mathrm{m}^3$
	\end{tasks}

	\item  从2至8的7个整数中随机取2个不同的数,则这两个数互质的概率为
	\begin{tasks}(4)
	\task $\frac{1}{6}$ \task $\frac{1}{3}$ \task $\frac{1}{2}$ \task $\frac{2}{3}$
	\end{tasks}

	\item  记函数$f(x)=\sin\left( \omega x+\frac{\pi}{4}\right)+b(\omega >0)$的最小正周期为$T$. 若$\frac{2\pi}{3}<T<\pi$,且$y=f(x)$的图像关于点$\left(\frac{3\pi}{2},2\right)$中心对称,则$f\left(\frac{\pi}{2}\right) = $
	\begin{tasks}(4)
	\task $1$ \task $\frac{3}{2} $ \task $\displaystyle\frac{5}{2}  $ \task $3$
	\end{tasks}

	\item  设$a=0.1\mathrm{e}^{0.1}$,$b=\frac{1}{9}$,$c=-\ln 0.9$,则
	\begin{tasks}(4)
	\task $a<b<c$ \task $c<b<a$ \task $c<a<b$ \task $a<c<b$
	\end{tasks}
	\item  已知正四棱锥的侧棱长为$l$,其各顶点都在同一球面上. 若该球的体积为$36\pi$,且$3\leqslant l\leqslant 3\sqrt{3}$,则该正四棱锥体积的取值范围是
	\begin{tasks}(4)
	\task $[18,\frac{81}{4}]$ \task $[\frac{27}{4},\frac{81}{4}]$ \task $[\frac{27}{4},\frac{64}{3}]$ \task $\left[18,27\right]$
	\end{tasks}
	
\end{enumerate}

\section{选择题:本大题共4小题,每小题5分,共20分。在每小题给出的四个选项中,有多项是符合题目要求的。全部选对得5分,部分选对得2分,有选错的得0分。}

\begin{enumerate}[itemsep=0.3em,topsep=0pt]
	\setcounter{enumi}{8}
	\item  已知正方体$ABCD-A_1B_1C_1D_1$,则
	\begin{tasks}(2)
	\task 直线$BC_1$与$DA_1$所成的角为$90^{\circ}$ 
	\task 直线$BC_1$与$CA_1$所成的角为$90^{\circ}$ 
	\task 直线$BC_1$与平面$BB_1D_1D$所成的角为$45^{\circ}$ 
	\task 直线$BC_1$与平面$ABCD$所成的角为$45^{\circ}$ 
	\end{tasks}

   \item  已知函数$f(x)=x^3-x+1$,则
	\begin{tasks}(2)
	\task $f(x)$有两个极值点 \task $f(x)$有三个零点
	\task 点$(0,1)$是曲线$y=f(x)$的对称中心 \task 直线$y=2x$是曲线$y=f(x)$的切线
	\end{tasks}

   \item  已知$O$为坐标原点,点$(1,1)$在抛物线$C:x^2=2py(p>0)$上,过点$B(0,-1)$的直线交$C$于$P$,$Q$两点,则
   \begin{tasks}(2)
   \task $C$的准线为$y=-1$
   \task 直线$AB$与$C$相切
   \task $|OP|\cdot|OQ|>|OA|^2$
   \task $|BP|\cdot|BQ|>|BA|^2$
   \end{tasks}

   \item 已知函数$f(x)$及其导函数$f^{\prime}(x)$的定义域均为$\mathbf{R}$,记$g(x)=f^{\prime}(x)$. 若$f(\frac{3}{2}-2x)$,$g(2+x)$均为偶函数,则
	\begin{tasks}(4)
		\task $f(0)=0$ \task $g(-\frac{1}{2})=0$ \task $f(-1)=f(-4)$ \task $g(-1)=g(2)$
	\end{tasks}

\end{enumerate}

\section{填空题:本题共4小题,每小题5分,共20分。}
\begin{enumerate}[itemsep=0.3em,topsep=0pt,resume]%\setcounter{enumi}{12}

\item  $(1-\frac{y}{x})(x+y)^8$的展开式中$x^2y^6$的系数为\blank{}(用数字作答).
\item  写出与圆$x^2+y^2=1$和$(x-3)^2+(y-4)^2=16$都相切的一条直线的方程\blank{}.
\item  若曲线$y=(x+a)e^x$有两条过坐标原点的切线,则$a$的取值范围是\blank{}.
\item  已知椭圆$C:\frac{x^2}{a^2}+\frac{y^2}{b^2}=1(a>b>0)$,$C$的上顶点为$A$,两个焦点为$F_1$,$F_2$,离心率为$\frac{1}{2}$. 过点$F_1$且垂直于$AF_2$的直线与$C$交于$D$,$E$两点,$|DE|=6$,则$\triangle ADE$的周长是\blank{}.

\end{enumerate}

\section{解答题:本题共6小题,共70分。解答应写出文字说明、证明过程或演算步骤。}

\begin{enumerate}[itemsep=0.5em,topsep=5pt,resume]%\setcounter{enumi}{17}
\item (10分)\\
记$S_n$为数列$\left\{a_{n}\right\}$的前$n$项和,已知$a_1=1$,$\{\frac{S_n}{a_n}\}$是公差为$\frac{1}{3}$的等差数列.

	\begin{enumerate}[itemsep=-0.3em,label={(\arabic*)},topsep=0pt,labelsep=.5em,leftmargin=3em]
	\item 求$\left\{a_{n}\right\}$的通项公式;
	\item 证明:$\frac{1}{a_1}+\frac{1}{a_2}+\cdots+\frac{1}{a_n}<2$.
\end{enumerate}

\item (12分)\\
记$\triangle ABC$的内角$A$,$B$,$C$的对边分别为$a$,$b$,$c$,已知$\frac{\cos A}{1+\sin A}=\frac{\sin 2B}{1+\cos 2B}$.
\begin{enumerate}[itemsep=-0.3em,label={(\arabic*)},topsep=0pt,labelsep=.5em,leftmargin=3em]
	\item 若$C=\frac{2\pi}{3}$,求$B$;\vspace{1em}
	\item 求$\frac{a^2+b^2}{c^2}$的最小值.
\end{enumerate}

\begin{minipage}[h]{.3\textwidth}
\item (12分)\\
如图,直三棱柱$ABC-A_1B_1C_1$的体积为4,$\triangle A_1BC$的面积为$2\sqrt{2}$.
\begin{enumerate}[itemsep=-0.3em,label={(\arabic*)},topsep=0pt,labelsep=.5em,leftmargin=3em]
	\item 求$A$到平面$A_1BC$的距离;
	\item 设$D$为$A_1C$的中点,$AA_1=AB$,平面$A_1BC\perp$平面$ABB_1A_1$,求二面角$A-BD-C$的正弦值.
\end{enumerate}
\end{minipage}\hspace{0.5em}
\begin{minipage}[h]{.2\textwidth}
\begin{figure}[H]
    \includegraphics[width=6cm]{19titu.png}	
\end{figure}
\end{minipage}\vspace{1em}

\item (12分)\\
一医疗团队为研究某地的一种地方性疾病与当地居民的卫生习惯(卫生习惯分为良好和不够良好两类)的关系,在已患该疾病的病例中随机调查了100例(称为病例组),同时在未患该疾病的人群中随机调查了100人(称为对照组),得到如下数据:
\begin{table}[H]
	\centering
	\setlength{\tabcolsep}{6mm}{
	\begin{tabular}{|c|c|c|}
	\hline
	& 不够良好 & 良好 \\ \hline
	病例组     &   40    &    60    \\ \hline
	对照组     &   10    &    90   \\ \hline
	\end{tabular}}
\end{table}
\begin{enumerate}[itemsep=-0.3em,label={(\arabic*)},topsep=0pt,labelsep=.5em,leftmargin=3em]
	\item 能否有99\%的把握认为患该疾病群体与未患该疾病群体的卫生习惯有差异?
 	\item 从该地的人群中任选一人,$A$表示事件“选到的人卫生习惯不够良好”,$B$表示事件“选到的人患有该疾病”,$\frac{P(B\mid A)}{P(\overline{B}\mid A)}$与$\frac{P(B\mid \overline{A})}{P(\overline{B}\mid \overline{A})}$的比值是卫生习惯不够良好对患该疾病风险程度的一项度量指标,记该指标为$R$.
 	\begin{itemize}
	    \item[($\mathrm{i}$)] 证明:$R=\frac{P(A\mid B)}{P(\overline{A}\mid B)}\cdot\frac{P(\overline{A}\mid \overline{B})}{P(A\mid \overline{B})}$;
	    \item[($\mathrm{ii}$)] 利用该调查数据,给出$P(A\mid B)$,$P(A\mid \overline{B})$的估计值,并利用(i)的结果给出$R$的估计值.
	\end{itemize}
	\textbf{附:}$K^2 = \frac{n(ab-bc)^2}{(a+b)(c+d)(a+c)(b+d)}$,
	\begin{tabular}{c|ccc}
		$P\left(K^2\geq k\right)$ & 0.050 & 0.010 & 0.001 \\ \hline
		$k$  & 3.841  & 6.635 & 10.828 \\ 
	\end{tabular}
\end{enumerate}

\item (12分) \\
已知点$A(2,1)$在双曲线$C:\frac{x^2}{a^2}-\frac{y^2}{a^2-1}=1(a>1)$上,直线$l$交$C$于$P$,$Q$两点,直线$AP$,$AQ$的斜率之和为0.
\begin{enumerate}[itemsep=-0.3em,label={(\arabic*)},topsep=0pt,labelsep=.5em,leftmargin=3em]
	\item 求$l$的斜率;
	\item 若$\tan\angle PAQ=2\sqrt{2}$,求$\triangle PAQ$的面积.
\end{enumerate}

\item (12分)\\
已知函数$ f(x)=e^x-ax$和$g(x)=ax-\ln x$有相同的最小值.
\begin{enumerate}[itemsep=-0.3em,label={(\arabic*)},topsep=0pt,labelsep=.5em,leftmargin=3em]
	\item 求$a$;
	\item 证明:存在直线$y=b$,其与两条曲线$y=f(x)$和$y=g(x)$共有三个不同的交点,并且从左到右的三个交点的横坐标成等差数列.
\end{enumerate}

\end{enumerate}


%%%%%%%%%%%%%%%%%%%%%%%%%%%%%%%%%%%%%%%%%%%%%%%%%%%%%%%%%%%%%%%%%%%%%%%%%%%%%%%
%------------------------------------结束--------------------------------------
%%%%%%%%%%%%%%%%%%%%%%%%%%%%%%%%%%%%%%%%%%%%%%%%%%%%%%%%%%%%%%%%%%%%%%%%%%%%%%%
\clearpage	
	
\end{document}
